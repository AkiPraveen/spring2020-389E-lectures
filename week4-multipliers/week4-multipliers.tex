\documentclass{beamer}
\usetheme{AnnArbor}
\usecolortheme{spruce}
\usepackage{circuitikz}
\usepackage{graphicx}

\title{Multipliers, Iterative Algorithms, Negative Numbers}
\subtitle{Leveraging Adders for a Greater Purpose}
\author[A Praveen \& A Krishnan]{Akilesh Praveen \& Ashwath Krishnan}
\institute{UMD}
\date{\today}



\begin{document}

    % title page
    \begin{frame}
        \titlepage
    \end{frame}
    
    % table of contents
    \begin{frame}
        \frametitle{Agenda}
        \tableofcontents
    \end{frame}
    
    \section{Announcements}
    
        \begin{frame}
                \vfill
                \centering
                \begin{beamercolorbox}[sep=8pt,center,shadow=true,rounded=true]{title}
                    \usebeamerfont{title}Announcements\par%
                \end{beamercolorbox}
                \vfill
             \end{frame}
    
        \subsection{Project 3}
        
            
            
            \begin{frame}
                \frametitle{Project 3}
                \begin{itemize}
                    \item Project 3's been released, and the name of the game is multipliers.
                    \item You'll find everything you need under the 'week 4' section on the course website
                    \item Today's lecture will give you all the background knowledge that you need to know about implementing a multiplier
                    
                    \item More info to come at the end of lecture
                    
                \end{itemize}
            \end{frame}
            
            \begin{frame}
            	\frametitle{Some Reminders}
            	\begin{itemize}
            		\item As the projects become more involved, we'd like to remind you to reach out and attend office hours if you are struggling.
                    \item Additionally, here's a reminder that all projects can be group projects, but you'll have to let us know \textbf{4 days} prior to the project's due date if you'll be working in a group. (Max size of 3)
            	\end{itemize}
            	
            \end{frame}
        
    \section{Multipliers}
    
    	\begin{frame}
                \vfill
                \centering
                \begin{beamercolorbox}[sep=8pt,center,shadow=true,rounded=true]{title}
                    \usebeamerfont{title}Multipliers\par%
                \end{beamercolorbox}
                \vfill
             \end{frame}
    
    	\subsection{Multiplication in Binary}
    
    	\begin{frame}
    		\frametitle{A Question}
    		\begin{center}
    			{\Large How does multiplication work in Binary?}
    		\end{center}
    	\end{frame}
    	
    	\begin{frame}
    		\frametitle{Multiplication Table}
    		\begin{itemize}
    			\item Multiplication works the same way in binary as it does in base-10!
    			\item Just like single digit multiplication in base-10 goes up to $ 9 \times 9 $, single digit multiplication in binary goes up to $ 1 \times 1$
    			\item Here's the basic multiplication table for binary:

    		\end{itemize}
    		
    		\centering
    		{\Large
    		\begin{tabular}{ |c|c|c| }
					\hline
				 	 & \textbf{0} & \textbf{1} \\ 
				 	\hline
				 	\textbf{0} & 0 & 0 \\  
				 	\hline
				 	\textbf{1} & 0 & 1 \\
				 	\hline 
					\end{tabular}
					}
    	\end{frame}
    	
    	
		
		
    
    
\end{document}
