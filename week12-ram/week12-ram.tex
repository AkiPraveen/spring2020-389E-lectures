\documentclass{beamer}
\usetheme{AnnArbor}
\usecolortheme{spruce}
\usepackage{circuitikz}
\usepackage{graphicx}
\usepackage{verbatim}

\title{RAM}
\subtitle{Really Awesome Memory}
\author[CMSC389E]{Akilesh Praveen | CMSC398E}
\institute{UMD}
\date{\today}

\begin{document}

    % title page
    \begin{frame}
        \titlepage
    \end{frame}
    
    % table of contents
    \begin{frame}
        \frametitle{Agenda}
        \tableofcontents
    \end{frame}
    
    \section{Announcements}
    
        \begin{frame}
                \vfill
                \centering
                \begin{beamercolorbox}[sep=8pt,center,shadow=true,rounded=true]{title}
                    \usebeamerfont{title}Announcements\par%
                \end{beamercolorbox}
                \vfill
             \end{frame}
    
        \subsection{Projects 5, 6, and 7}
        
            
            
            \begin{frame}
                \frametitle{Projects 5, 6, 7}
                \begin{itemize}
                    \item Projects 5, 6, and 7 are now released on Piazza
                    \item Relevant instructional material is/will be linked
                    \item They can be done in \textbf{any order}, but I would suggest doing them in order (5, then 6, then 7)
                    \item We already did a lecture on Project 5 and 6, today we'll be talking about \textbf{Project 7}
                    
                \end{itemize}
            \end{frame}
            
            
    \section{Intro + Background}
    
    	\begin{frame}
                \vfill
                \centering
                \begin{beamercolorbox}[sep=8pt,center,shadow=true,rounded=true]{title}
                    \usebeamerfont{title}Intro\par%
                \end{beamercolorbox}
                \vfill
             \end{frame}
    
    		\begin{frame}
    			\frametitle{Intro}
    			\begin{itemize}
    				\item We've built the ALU; the brains of the operation
    				\item Now we need a few more things to take this from just a calculator circuit to an actual computer
    				\begin{itemize}
    					\item Ways to \textbf{store} programs
    					\item Ways to \textbf{interpret} those programs
    					\item Ways to \textbf{execute} those programs
    					\item Ways to \textbf{store data} for those programs while they're executing
    				\end{itemize}
    				\item We're going to use the digital logic circuit theory to build circuits to address all of these! (Projects 5, 6, and 7)
    			\end{itemize}
    		\end{frame}
    		
    		\begin{frame}
    			\frametitle{Intro}
    			
    				\begin{itemize}
    					\item Ways to \textbf{store} programs - \textbf{ROM} \textit{(Project 5)}
    					\item Ways to \textbf{interpret} those programs - \textbf{389E Assembly} \textit{(Project 5)}
    					\item Ways to \textbf{execute} those programs - \textbf{Program Counter} \textit{(Project 6)}
    					\item Ways to \textbf{store data} for those programs while they're executing - \textbf{RAM} \textit{(Project 7)}
    					\item Today, we'll be talking about ways to store data for these programs, using registers of \textbf{RAM}.
    				\end{itemize}
    				
    			
    		\end{frame}
    		
    		
		
            
        
   	
\end{document}
