\documentclass{beamer}
\usetheme{AnnArbor}
\usecolortheme{spruce}
\usepackage{circuitikz}

\title{Logic Gates}
\subtitle{The Building Blocks of Digital Logic in Redstone}
\author[A Praveen]{Akilesh Praveen - CMSC389E}
\institute{UMD}
\date{\today}

\begin{document}

    % title page
    \begin{frame}
        \titlepage
    \end{frame}
    
    % table of contents
    \begin{frame}
        \frametitle{Agenda}
        \tableofcontents
    \end{frame}
    
    \section{Announcements}
    
        \begin{frame}
                \vfill
                \centering
                \begin{beamercolorbox}[sep=8pt,center,shadow=true,rounded=true]{title}
                    \usebeamerfont{title}Announcements\par%
                \end{beamercolorbox}
                \vfill
             \end{frame}
    
        \subsection{Project 1}
        
            
            
            \begin{frame}
                \frametitle{Project 1}
                \begin{itemize}
                    \item Project 1 has been released, and can be found on the course website under 'Week 2' on the calendar.
                    \item Everything you need to complete the project will be covered in today's lecture, with optional supplementary material provided on the course website under Week 2's resources section.
                \end{itemize}
            \end{frame}
        
            
    \section{Logic Gates}
        \subsection{Inputs \& Outputs}
        
            \begin{frame}
                \vfill
                \centering
                \begin{beamercolorbox}[sep=8pt,center,shadow=true,rounded=true]{title}
                    \usebeamerfont{title}Inputs and Outputs\par%
                \end{beamercolorbox}
                \vfill
             \end{frame}
        
        
            \begin{frame}
                \frametitle{Inputs \& Outputs}
                \begin{itemize}
                    \item The Basic Logic Gates make up all of our circuits
                    \item More advanced gates are simply just combinations of the four types of basic gates
                    \item The gates all take two inputs, and produce one output (with an exception of \texttt{Not}, which only takes one input)
                \end{itemize}
            \end{frame}
            
            \begin{frame}
                \frametitle{Inputs \& Outputs in Redstone}
                \begin{itemize}
                    \item Logic Gates are the absolute most basic constructs of the redstone circuits we will be building in this class
                    \item Let's try and represent them using the redstone mechanics available to us
                    \item Remember: we have torches, wires, and repeaters
                \end{itemize}
            \end{frame}
            
            \begin{frame}
                \frametitle{Inputs \& Outputs in Redstone}
                \begin{itemize}
                    \item How can we take our understanding of this and translate it into Redstone?
                    \item Let's go step by step and draw some comparisons
                    \item First, let's handle the input and output for our logic gates
                    \begin{itemize}
                        \item Input for a logic gate is composed of 1 or 2 True or False values in discrete math, so let's call it 1 or 2 redstone wires carrying either current or no current
                        \item Output for a logic gate is a True or False value in discrete math, so let's say our output value can be represented by a redstone wire carrying either current or no current
                    \end{itemize}
                    \item Think about what else you'd need to create these gates in Minecraft while we review them
                \end{itemize}
            \end{frame}
            
        \subsection{The Basic Gates}
        
        
            
            \begin{frame}
                \vfill
                \centering
                \begin{beamercolorbox}[sep=8pt,center,shadow=true,rounded=true]{title}
                    \usebeamerfont{title}The Basic Gates\par%
                \end{beamercolorbox}
                \vfill
             \end{frame}
            
        
            \begin{frame}
                \frametitle{Not}
                \begin{itemize}
                    \item This gate is not not not not not not one of the most useful logic gates!
                    \item Note that this is the only gate with one input
                    \item Often used in conjunction with other gates to produce a 'not' version of that gate (e.g. \texttt{AND} and \texttt{NAND})
                    \item To produce this gate in Minecraft, we have to make use of a special redstone mechanic involving torches\newline
                \end{itemize}
                
                
                % the following section is the logic gate in circuitikz and the truth table
                
                \centering
                
                \begin{columns}
                    
                    \column{0.5\textwidth}
                    
                    \centering
                
                    \begin{circuitikz} \draw
                    node[not port]{};
                    \end{circuitikz}
                    
                    \column{0.5\textwidth}
                    
                    
                    
                    \begin{tabular}{ |p{1cm}||p{1cm}|}
                     \hline
                     \multicolumn{2}{|c|}{Not} \\
                     \hline
                     In & Out\\
                     \hline
                     T & F\\
                     F & T\\
                     \hline
                    \end{tabular}
                    
                \end{columns}
                
            \end{frame}
            
            
            \begin{frame}
                \frametitle{And \& its evil twin, Nand}
                \begin{itemize}
                    \item Best way to think of this is to think of the equivalent in English
                    \item If input \texttt{a} AND input \texttt{b} are true, produce a true output 
                    \item Placing a \texttt{NOT} gate after an \texttt{AND} gate results in a \texttt{NAND} gate
                    \item We are explicitly mentioning the \texttt{NAND} gate for a reason, and we will return to this idea later\newline
                \end{itemize}
                
                
                % the following section is the logic gate in circuitikz and the truth table
                
                \centering
                
                \begin{columns}
                    
                    \column{0.25\textwidth}
                    
                    \centering
                
                    \begin{circuitikz} \draw
                    node[and port]{};
                    \end{circuitikz}
                    
                    \column{0.25\textwidth}
                    
                    
                    
                    \begin{tabular}{ |p{1cm}||p{1cm}|}
                     \hline
                     \multicolumn{2}{|c|}{And} \\
                     \hline
                     In & Out\\
                     \hline
                     T F & F\\
                     F T & F\\
                     T T & T\\
                     F F & F\\
                     \hline
                    \end{tabular}
                    
                    \column{0.25\textwidth}
                    
                    \centering
                
                    \begin{circuitikz} \draw
                    node[nand port]{};
                    \end{circuitikz}
                    
                    \column{0.25\textwidth}
                    \centering
                    
                    \begin{tabular}{ |p{1cm}||p{1cm}|}
                     \hline
                     \multicolumn{2}{|c|}{Nand} \\
                     \hline
                     In & Out\\
                     \hline
                     T F & T\\
                     F T & T\\
                     T T & F\\
                     F F & T\\
                     \hline
                    \end{tabular}
                    
                \end{columns}
                
            \end{frame}
            
            \begin{frame}
                \frametitle{Or \& its evil twin, Nor}
                \begin{itemize}
                    \item Very similar to \texttt{AND} in that the English word defines its behavior exactly
                    \item If input \texttt{a} OR input \texttt{b} are true (or both), produce a true output 
                    \item Placing a \texttt{NOT} gate after an \texttt{OR} gate results in a \texttt{NOR} gate\newline
                \end{itemize}
                
                
                % the following section is the logic gate in circuitikz and the truth table
                
                \centering
                
                \begin{columns}
                    
                    \column{0.25\textwidth}
                    
                    \centering
                
                    \begin{circuitikz} \draw
                    node[or port]{};
                    \end{circuitikz}
                    
                    \column{0.25\textwidth}
                    
                    
                    
                    \begin{tabular}{ |p{1cm}||p{1cm}|}
                     \hline
                     \multicolumn{2}{|c|}{Or} \\
                     \hline
                     In & Out\\
                     \hline
                     T F & T\\
                     F T & T\\
                     T T & T\\
                     F F & F\\
                     \hline
                    \end{tabular}
                    
                    \column{0.25\textwidth}
                    
                    \centering
                
                    \begin{circuitikz} \draw
                    node[nor port]{};
                    \end{circuitikz}
                    
                    \column{0.25\textwidth}
                    \centering
                    
                    \begin{tabular}{ |p{1cm}||p{1cm}|}
                     \hline
                     \multicolumn{2}{|c|}{Nor} \\
                     \hline
                     In & Out\\
                     \hline
                     T F & F\\
                     F T & F\\
                     T T & F\\
                     F F & T\\
                     \hline
                    \end{tabular}
                    
                \end{columns}
                
            \end{frame}
            
            
            \begin{frame}
                \frametitle{Xor, the Unique Gate}
                \begin{itemize}
                    \item The \texttt{XOR} gate's name an abbreviated form of 'exclusive or'
                    \item An easy way to think about this is: If both inputs are unique, this gate produces \texttt{TRUE}. Otherwise, it produces \texttt{FALSE} 
                    \item The logical complement of the \texttt{XOR} gate is the \texttt{XNOR} gate, which has the opposite truth table.
                    \item For simplicity's sake, we won't focus heavily on \texttt{XNOR} for now.\newline
                \end{itemize}
                
                
                % the following section is the logic gate in circuitikz and the truth table
                
                \centering
                
                \begin{columns}
                    
                    \column{0.5\textwidth}
                    
                    \centering
                
                    \begin{circuitikz} \draw
                    node[xor port]{};
                    \end{circuitikz}
                    
                    \column{0.5\textwidth}
                    
                    
                    
                    \begin{tabular}{ |p{1cm}||p{1cm}|}
                     \hline
                     \multicolumn{2}{|c|}{Xor} \\
                     \hline
                     In & Out\\
                     \hline
                     T F & T\\
                     F T & T\\
                     T T & F\\
                     F F & F\\
                     \hline
                    \end{tabular}
                    
                    
                    
                \end{columns}
                
            \end{frame}
            
             \begin{frame}
                \frametitle{Logical Complements}
                \begin{itemize}
                    \item Let's talk about logical complements- you've just seen a few
                    \item Occasionally, we'll have a gate or circuit that produces the exact opposite of what we want
                        \begin{itemize}
                            \item I'm saying 'circuit' for a reason
                        \end{itemize}
                    \item In this case, you'll want the logical complement of whatever gate/circuit you've got
                    \item When drawing circuit diagrams, using the 'dot' notation is sometimes less cumbersome than just sprinkling \texttt{NOT} gates everywhere
                \end{itemize}
               
            \end{frame}
            
            \begin{frame}
                \frametitle{A Small Exercise}
                \begin{itemize}
                    \item Most of you have taken/are taking 250, so let's make your TAs proud!
                    \item Try to construct an \texttt{XOR} gate using only \texttt{OR}s, \texttt{AND}s, and \texttt{NOT}s
                        \begin{itemize}
                            \item Try to be as efficient as possible- circuits had to be made by hand in the old days, and every piece had a cost associated with it!
                            \item It's also worth noting that more gates = more build time in Minecraft, so it's good to learn to think efficiently in this class
                        \end{itemize}
                    \item Why is this important?
                \end{itemize}
               
            \end{frame}
            
            
        \subsection{Applications in Computing}
        
            \begin{frame}
                \vfill
                \centering
                \begin{beamercolorbox}[sep=8pt,center,shadow=true,rounded=true]{title}
                    \usebeamerfont{title}Applications in Computing\par%
                \end{beamercolorbox}
                \vfill
            \end{frame}
            
            \begin{frame}
                \frametitle{The \texttt{NAND} Gate's Importance}
                \begin{itemize}
                    \item Let's talk about the \texttt{NAND} Gate
                    \item Why are they the most commonly found gates in computers?
                    \item A Few Reasons
                    \begin{itemize}
                        \item \texttt{NAND} is what we call '\textbf{functionally complete}'. (So is the \texttt{NOR} gate)
                        \item If a gate is \textbf{functionally complete}, all other gates can be constructed from it
                        \item in CMOS, (a common circuit fabrication process), \texttt{NAND} is smaller and faster than a \texttt{NOR} gate
                    \end{itemize}
                    \item TL;DR - Due to already set industry precedents and the initial ease to manufacture them, \texttt{NAND} gates were the pick over \texttt{NOR} gates for the universal gate that engineers used to build other gates, and eventually other circuits out of.
                \end{itemize}
               
            \end{frame}
            
            \begin{frame}
                \frametitle{The \texttt{NAND} Gate's Importance (cont'd)}
                \begin{itemize}
                    \item It turns out, we are able to create a \texttt{NAND} gate in Minecraft.
                    \item Because of this, we can conclude that we can create a \textbf{functionally complete} gate in Minecraft.
                    \item Therefore, we have proven that we can implement all digital logic in Minecraft.
                    \item Does this mean we will build everything in terms of NAND gates in Minecraft?
                    \begin{itemize}
                        \item Absolutely not. Transistors and hardware are one thing in real life, but very different in Minecraft. There are better, easier ways to make gates in Minecraft.
                        \item In this course, it's important to make note of the few, but key differences between hardware in real life and hardware in game
                    \end{itemize}
                \end{itemize}
               
            \end{frame}
            
            \begin{frame}
                \frametitle{The \texttt{NAND} Gate's Importance (cont'd)}
                \begin{itemize}
                    \item Let's reinforce the idea of functional completeness
                    \item For example, here's how we can make the \texttt{NOT} gate using only \texttt{NAND} gates.
                    \item This actually turns out to be the simplest of all once we give it a try. \newline
                \end{itemize}
                
                \begin{columns}
                    
                    \column{0.5\textwidth}
                    
                    \centering
                
                    \begin{circuitikz} \draw
                    (1,1) node[](a) {A}
                    (3.25,1) node[](out) {OUT}
                    (2,1) node[not port](mynot){};
                    \end{circuitikz}
                    
                    \column{0.5\textwidth}
                    
                    \centering
                    
                    \begin{circuitikz} \draw
                    (0,1.5) node[](a) {A}
					(0,0.5) node[](a2) {A}                  
                    (2,1) node[nand port] (mynand){}
                    (2.75,1) node[] (c) {OUT}
                    (a) -| (mynand.in 1)
                    (a2) -| (mynand.in 2);
                    \end{circuitikz}
                    
                    
                \end{columns}
                
               
            \end{frame}
            
            \begin{frame}
                \frametitle{The \texttt{NAND} Gate's Importance (cont'd)}
                \begin{itemize}
                    \item What about the \texttt{AND} gate?
                    \item Try constructing an \texttt{AND} gate using only \texttt{NAND} gates. 
                    \item Remember, you want to think of it more like a circuit- ask yourself: How can I produce a circuit that takes input and produces output identical to a single \texttt{AND} gate using only \texttt{NAND} gates? \newline
                \end{itemize}
                
                \begin{columns}
                    
                    \column{0.5\textwidth}
                    
                    \centering
                
                    \begin{circuitikz} \draw
                    (0,1.5) node[](a) {A}
					(0,0.5) node[](b) {B}                  
                    (2,1) node[and port] (myand){}
                    (2.75,1) node[] (c) {OUT}
                    (a) -| (myand.in 1)
                    (b) -| (myand.in 2);
                    \end{circuitikz}
                    
                    \column{0.5\textwidth}
                    
                    \centering
                    
                    \begin{circuitikz} \draw
                    (-1.75,0.25) node[](a) {A}
					(-1.75,-0.25) node[](b) {B}
					(2.75,0) node[] (c) {OUT}
                    (0,0) node[nand port] (mynand1) {}
                    (2,0) node[nand port] (mynand2) {}
                    (mynand1.out) -| (mynand2.in 1)
                    (mynand1.out) -| (mynand2.in 2);
                    \end{circuitikz}
                    
                    
                \end{columns}
                
               
            \end{frame}
            
            \begin{frame}
                \frametitle{The \texttt{NAND} Gate's Importance (cont'd)}
                \begin{itemize}
                    \item And the \texttt{OR} gate?
                    \item Try constructing an \texttt{OR} gate using only \texttt{NAND} gates. 
                    \item Again, think of it like a circuit. Given limited tools, try and solve the boolean equation that we've posed for you. \newline
                \end{itemize}
                
                \begin{columns}
                    
                    \column{0.5\textwidth}
                    
                    \centering
                
                    \begin{circuitikz} \draw
                    (0,1.5) node[](a) {A}
					(0,0.5) node[](b) {B}                  
                    (2,1) node[or port] (myor){}
                    (2.75,1) node[] (c) {OUT}
                    (a) -| (myor.in 1)
                    (b) -| (myor.in 2);
                    \end{circuitikz}
                    
                    \column{0.5\textwidth}
                    
                    \centering
                    
                    \begin{circuitikz} \draw
                    (-1.5,0) node[](b) {B}
                    (-1.5,2) node[](a) {A}
                    (2.75,1) node[](out) {OUT}
                    (0.5,2) node[nand port] (mynand0) {}
                    (0.5,0) node[nand port] (mynand1) {}
                    (2,1) node[nand port] (mynand2) {}
                    (b) -| (mynand1.in 1)
                    (b) -| (mynand1.in 2)
                    (a) -| (mynand0.in 1)
                    (a) -| (mynand0.in 2)
                    
                    (mynand0.out) -| (mynand2.in 1)
                    (mynand1.out) -| (mynand2.in 2);
                    \end{circuitikz}
                    
                    
                \end{columns}
                
               
            \end{frame}
            
            \begin{frame}
                \frametitle{The \texttt{NAND} Gate's Importance (cont'd)}
                \begin{itemize}
                    \item Finally, let's try the \texttt{XOR} gate
                    \item Try constructing an \texttt{XOR} gate using only \texttt{NAND} gates. 
                    \item This one's a little tougher- try and get creative, or just wait for us to reveal the solution. \newline
                \end{itemize}
                
                \begin{columns}
                    
                    \column{0.35\textwidth}
                    
                    \centering
                
                    \begin{circuitikz} \draw
					(0,1.5) node[](a) {A}
					(0,0.5) node[](b) {B}                  
                    (2,1) node[xor port] (myxor){}
                    (2.75,1) node[] (c) {OUT}
                    (a) -| (myxor.in 1)
                    (b) -| (myxor.in 2);
                    \end{circuitikz}
                    
                    \column{0.65\textwidth}
                    
                    \centering
                    
                    \begin{circuitikz} \draw
                    (0,1) node[nand port] (mynand4) {}
                    (2,2) node[nand port] (mynand0) {}
                    (2,0) node[nand port] (mynand1) {}
                    (4,1) node[nand port] (mynand2) {}
                    (4.75,1) node[] (c) {OUT}
                    (-2, 2.5) node[](a) {A}
                    (-2,-0.5) node[](b) {B}
                    (a) -| (mynand4.in 1)
                    (a) -| (mynand0.in 1)
                    (b) -| (mynand4.in 2)
                    (b) -| (mynand0.in 2)
                    (mynand4.out) -| (mynand1.in 1)
                    (mynand4.out) -| (mynand0.in 2)
                    (mynand0.out) -| (mynand2.in 1)
                    (mynand1.out) -| (mynand2.in 2);
                    \end{circuitikz}
                    
                    
                \end{columns}
                
               
            \end{frame}
            
            
            
            
            
            
    \section{Note on Logic Rules / Efficiency}
    
    	 \begin{frame}
                \vfill
                \centering
                \begin{beamercolorbox}[sep=8pt,center,shadow=true,rounded=true]{title}
                    \usebeamerfont{title}Logic Rules \& Efficiency\par%
                \end{beamercolorbox}
            \vfill
         \end{frame}
    
    	\begin{frame}
                \frametitle{Logic Rules: In Review}
                \begin{itemize}
                    \item When discussing boolean logic, we can't forget about the basic rules, or equivalencies, that come along with it.
                    \item Refer to this handy cheatsheet: \href{http://www.cs.umd.edu/class/spring2019/cmsc250-020X/files/logic-laws.pdf}{\beamergotobutton{CMSC250 Equivalency Reference Sheet}}
                    \item These rules exist to help us simplify circuits (predicates) that are otherwise more cumbersome to deal with. (In our case, to implement in hardware/Minecraft)
                    \item When designing logical circuits, you will find it \textbf{infinitely} more easy if you consistently keep these rules in mind.
                \end{itemize}
            \end{frame}
            
            
         \begin{frame}
         	\frametitle{An Example}
         	\begin{itemize}
         		\item Here's a simple example of how using these logic rules can help us 'simplify' a circuit while retaining the same functionality
         		\item Let's say we have the given circuit. How would we systematically look at simplifying this?
         		
         	\end{itemize}
         	
         	\centering
                    
                    \begin{circuitikz} \draw
                    
                    (-2, 2.5) node[](a) {A}
                    (-2, 1.5) node[](b1) {B}
                    (-2, 0.5) node[](b2) {B}
                    (-2,-0.5) node[](c) {C}
                    (0,0) node[and port] (myand1) {}
                    (0,2) node[and port] (myand2) {}
                    (2,1) node[or port] (myor1) {}
                    (2.75,1) node[] (out) {OUT}
                    (a) -| (myand2.in 1)
                    (b1) -| (myand2.in 2)
                    (b2) -| (myand1.in 1)
                    (c) -| (myand1.in 2)
                    (myand2.out) -| (myor1.in 1)
                    (myand1.out) -| (myor1.in 2);
                    \end{circuitikz}
         \end{frame}
         
         \begin{frame}
         	\frametitle{An Example (cont'd)}
			
			\centering
			\begin{circuitikz} \draw
                    
                    (-2, 2.5) node[](a) {A}
                    (-2, 1.5) node[](b1) {B}
                    (-2, 0.5) node[](b2) {B}
                    (-2,-0.5) node[](c) {C}
                    (0,0) node[and port] (myand1) {}
                    (0,2) node[and port] (myand2) {}
                    (2,1) node[or port] (myor1) {}
                    (2.75,1) node[] (out) {OUT}
                    (a) -| (myand2.in 1)
                    (b1) -| (myand2.in 2)
                    (b2) -| (myand1.in 1)
                    (c) -| (myand1.in 2)
                    (myand2.out) -| (myor1.in 1)
                    (myand1.out) -| (myor1.in 2);
                    \end{circuitikz}         	
         	
         	\begin{itemize}
         		\item Let's take a look at the logical rules presented to us.
         		\item \href{http://www.cs.umd.edu/class/spring2019/cmsc250-020X/files/logic-laws.pdf}{\beamergotobutton{CMSC250 Equivalency Reference Sheet}}
         		\item Tip: Write out the boolean logic equivalent of this circuit.
         		\item $ ( A \land B ) \lor (B \land C) $
         		\item Look familiar? Which rule can we apply to simplify our circuit?
         	\end{itemize}
         \end{frame}
            
         \begin{frame}
         	\frametitle{An Example (cont'd)}
         	
         	\begin{itemize}
         		\item After applying the distributive law, we can see that our previous statement simplifies to $ B \land (A \lor C) $
         		\item As a result, we can represent it as a simpler logical circuit, as you'll see below.
         		
         	\end{itemize}
         	
         	\centering
			\begin{circuitikz} \draw
                    
                    (-2, 2.5) node[](a) {A}
                    (-2, 1.5) node[](c) {C}
                    (-2, 0.5) node[](b1) {B}
                    
                    (0,2) node[or port] (myor) {}
                    (2,1) node[and port] (myor1) {}
                    (2.75,1) node[] (out) {OUT}
                    (a) -| (myor.in 1)
                    (c) -| (myor.in 2)
                    (myor.out) -| (myor1.in 1)
                    (b1) -| (myor1.in 2);
                    \end{circuitikz}       
         	
         \end{frame}
         
         \begin{frame}
         	\frametitle{Why learn this? (Takeaways)}
         	\begin{itemize}
         		\item Knowing the basics of boolean logic opens the doors for us to make more advanced circuits than just simple truth table representations.
         		\item Understanding \texttt{NAND} gates and \texttt{NOR} gates and functional completeness provides us with some background knowledge of digital logic design \& doubles as an excellent learning tool for circuit construction.
         		\item Boolean logic rules are essential to designing circuits in the most efficient way possible, saving time, money, and materials in the real world, and saving us from carpal tunnel in Minecraft.
         	\end{itemize}
         \end{frame}
         
   	\section{Project 1}
   		
   		\begin{frame}
                \vfill
                \centering
                \begin{beamercolorbox}[sep=8pt,center,shadow=true,rounded=true]{title}
                    \usebeamerfont{title}Project 1\par%
                \end{beamercolorbox}
                \vfill
             \end{frame}
             
        \subsection{What is an ALU?}
             
      	\begin{frame}
      		\frametitle{Project 1}
      		\begin{itemize}
      			\item For this project, you will be building the basis of an Arithmetic/Logic Unit
      			\item As the name implies, the ALU performs \textbf{arithmetic} and \textbf{logical} operations on binary inputs.
      			\item This seems like a pretty basic part of a computer, but all computers these days come with an ALU
      			\begin{itemize}
      				\item At a very basic level, computers still have to perform these simplistic operations. This is where it happens.
      			\end{itemize}
      			\item Further reading on ALUs is available on the course website
      		\end{itemize}
      	\end{frame}
      	
      	\begin{frame}
      		\frametitle{Project 1}
      		\begin{itemize}
      			\item For this project, you will be working on the logical part of an ALU
      			\item In simplest terms, you'll be given a few inputs:
      			\begin{itemize}
      				\item Some input signals (like A and B in the above examples) to perform a logical operation with
      				\item Output blocks to 'catch' the outputs of each of these signals
      			\end{itemize}
      			\item In order to do this efficiently (i.e. use as little wire as possible), you will be leveraging buses (not the 122 Green kind)
      			\item A video on buses in Minecraft can be found on the course website under Week 1 resources
      		\end{itemize}
      	\end{frame}
            
    
    
\end{document}
