\documentclass{beamer}
\usetheme{AnnArbor}
\usecolortheme{spruce}

\title{Introduction}
\subtitle{Welcome to Digital Logic Through Minecraft!}
\author[A Praveen \& A Krishnan]{Akilesh Praveen \& Ashwath Krishnan}
\institute{UMD}
\date{\today}

\begin{document}

    % title page
    \begin{frame}
        \titlepage
    \end{frame}
    
    % table of contents
    \begin{frame}
        \frametitle{Agenda}
        \tableofcontents
    \end{frame}
    
    \section{Announcements}
        \subsection{Buy Minecraft!}
            
            \begin{frame}
                \frametitle{Buy Minecraft!}
                \begin{itemize}
                    \item It is essential to this course that you buy the Java edition of Minecraft from Mojang's official website, located at \href{https://www.minecraft.net/en-us/store/minecraft-java-edition//}{https://www.minecraft.net/en-us/store/minecraft-java-edition}
                    \item We cannot facilitate the use of the Pocket Edition, Bedrock Edition, and Console Versions of Minecraft
                    \item At least it's cheaper than a textbook
                \end{itemize}
            \end{frame}
        
            
        \subsection{Project 0}
            
            \begin{frame}
                \frametitle{Project 0}
                \begin{itemize}
                    \item Project 0 is Tutorial Island, and is meant to get you up and running with Minecraft.
                    \item We encourage you to come to office hours to ensure that you are able to run and submit your project properly.
                    \item Don't worry too much, more on projects later.
                \end{itemize}
            \end{frame}
            
    \section{About the Course}
    
        \subsection{Instructors}
        
            \begin{frame}
                \frametitle{Instructors}
                \begin{columns}
                    \column{0.5\textwidth}
                    \centering
                    Akilesh Praveen
                    \column{0.5\textwidth}
                    \centering
                    Ashwath Krishnan
                    \end{columns}
            \end{frame}
            
        \subsection{What to Expect}
        
            \begin{frame}
                \frametitle{What to Expect}
                    \begin{itemize}
                        \item A continuation of logic gates from CMSC250
                        \item Or, implementations of what you learned in ENEE244
                        \item We will be building the logic structures that are the foundations of computer hardware
                        \item Starting off with logic gates and adders and eventually moving onto more advanced concepts such as latches and memory
                    \end{itemize}
            \end{frame}
            
            \begin{frame}
                \frametitle{Class Structure}
                    \begin{itemize}
                        \item All projects will be submitted as Minecraft worlds on the CS submit server (submit.cs.umd.edu)
                        \item One project per week
                            \begin{itemize}
                                \item Assigned Friday, due the next Monday
                                \item You are given a little over a week for each project
                            \end{itemize}
                        \item 13 total lectures
                        \item 1 in-class midterm
                        \item Cumulative final project
                    \end{itemize}
            \end{frame}
            
            \begin{frame}
                \frametitle{Class Structure}
                    \begin{itemize}
                        \item Theory taught in class during lecture, with extra video resources provided on the course website
                        \item Practice handed in as projects and graded
                        \item Extra credit! (Explained later)
                    \end{itemize}
            \end{frame}
            
            \begin{frame}
                \frametitle{Class Structure}
                    \begin{itemize}
                        \item You will be provided with sufficient theoretical background during lecture, supplemented by additional content posted on the course website if you wish to view it.
                        \item A project description will be uploaded to the course website each class, which will apply what you've learned.
                        \item You are permitted 2 unexcused absences, but are still required to do the project assigned that day in class.
                        \item See syllabus for more details
                    \end{itemize}
            \end{frame}
            
        \subsection{What to Get Familiar With}
        
            \begin{frame}
                \frametitle{What to Get Familiar With}
                    \begin{itemize}
                        \item Minecraft Redstone basics
                        \item Basic logic design schemes, symbols, truth tables
                        \item Submitting projects
                        \item Your classmates!
                        \item The Course Website- Extensive resources for all these topics can be found here (and more)
                    \end{itemize}
            \end{frame}
            
            \begin{frame}
                \frametitle{Participation}
                    \begin{itemize}
                        \item Graded through ELMS quizzes
                        \item Timed and available during class
                        \item Up to a possible 10\% of extra total participation points
                        \item Participation is the same as extra credit
                            \begin{itemize}
                                \item Even if you get a 0 for participation, you can still get 100\% in the class, but would you all really do that to me?
                            \end{itemize}
                        \item Now we've got the first quiz for you!
                        \item Usually, these quizzes are open till the end of class, but since it's the first one, this will remain open until midnight tonight.
                    \end{itemize}
            \end{frame}
            
            \begin{frame}
                \frametitle{Midterm}
                    \begin{itemize}
                        \item Based on projects and concepts taught in class
                        \item Date: TBD
                        \item In class, on paper
                    \end{itemize}
            \end{frame}
            
            \begin{frame}
                \frametitle{Projects}
                    \begin{itemize}
                        \item Assigned on Friday, based on the concepts covered in lecture.
                        \item Will be posted under their corresponding week on the course website
                        \item Due 10 days later, on Monday at 11:59PM
                        \item IMPORTANT! You need to submit a GFA for each project in order to pass!
                        \item All projects can be done with partners
                        \begin{itemize}
                            \item Check the 'partner project guidelines' link on the course website for instructions on how to properly submit partner projects
                        \end{itemize}
                        
                    \end{itemize}
            \end{frame}
            
            \begin{frame}
                \frametitle{Online Resources}
                    \begin{itemize}
                        \item The course website contains videos, cheat sheets, guides, and external links that provide extensive supplementary content for this course. It's highly recommended you take a look.
                        \item Q/A on Piazza
                            \begin{itemize}
                                \item Talk to one of us after class if you are unable to access the Piazza
                                \item We will also be posting updates on Piazza, so make sure to check it fairly often
                            \end{itemize}
                    \end{itemize}
            \end{frame}
            
            
    
    
\end{document}
